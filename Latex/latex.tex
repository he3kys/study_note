%导言区
\documentclass[10pt]{book}%normalsize 为10磅

\usepackage{ctex} %ctex用于解析中文

%自定义myfont命令,实现格式与内容分析
\newcommand{\myfont}{\textit{\textbf{\textsf{Fancy Text}}}}

% 标题
\title{杂谈勾股定理}
\author{张三}
\date{\today}

%正文区
\begin{document}
    \maketitle %输出标题

    %字体族设置(罗马字体,无衬线字体,打字机字体)
    \textrm{Roman Family} %罗马字体
    \textsf{Sans serif family} %无衬线字体
    \texttt{typewriter family} %打字机字体

    \rmfamily Roman Family %声明后面的字体为罗马字体,与textrm功能类似
    \sffamily sans serif family
    \ttfamily typewriter family

    % 字体系列设置
    \textmd{Medium Series}
    \textbf{Boldface Series}

    \mdseries medium Series
    \bfseries boldface Series


    %字体形状设置(直立,斜体,伪斜体,小型大写)
    \textup{Upright Shape}
    \textit{Italic Shape}
    \textsl{Slanted Shape}
    \textsc{Small Caps Shape}

    \upshape upright shape

    %中文字体
    {\songti 宋体} \quad {\heiti 黑体} \quad {\fangsong 仿宋} \quad {\kaishu 楷书} \quad  

    中文字体的\textbf{粗体}与\textit{斜体}

    %字体大小(相对于normalsize而言,normalsize在文档类中声明)
    {\tiny Hello}\\
    {\scriptsize Hello}\\
    {\footnotesize Hello}\\
    {\small Hello}\\
    {\normalsize Hello}\\
    {\Large Hello}\\
    {\LARGE Hello}\\
    {\huge Hello}\\
    {\Huge Hello}\\

    %中文字号设置
    \zihao{-0} 你好
    \zihao{5} 你好

    %自定义命令
    \myfont
\end{document}

